\documentclass[12pt]{article}
\usepackage{listings}
\title{Dynastic Score Calculation}
\author{King of Men}

\lstset{
breaklines=true,
keepspaces=true,
showspaces=false,
}

\begin{document}
\maketitle

\section{Introduction}

The dynastic score is calculated by the source code in the Appendix,
which is a complete description of the algorithm. However, the
auxiliary outline in English below may serve as a rough guide for
understanding the code.

\section{Base Score}

For each player dynasty \footnote{Stored in the \texttt{custom\_score} object of
the custom file specified in \texttt{config.txt}.}, the converter will
go through the characters (living and dead) of that dynasty, looking
for score-worthy traits and for titles. Traits are considered below.
For each landed title of County level and above, the converter goes
through the events associated with the title, that is to say, when it
changed hands and who it went to. For each holder of a player dynasty,
it calculates a title score, which is equal to the number of days the
dynast held the title times the title weight. The title weight is one for
counties, three for duchies, five for kingdoms, and seven for
Empires. In addition, creating the title counts as ten years of
holding it; the curse of Pharaoh and the GM is upon those who
would exploit this by repeatedly destroying and re-creating
titles. Note again that landless titles are ignored, so having the
world's greatest collection of rebels does not give score.

The total title score of the dynasty, i.e. the sum of the title scores
of all the dynasts, is the \emph{base score}. Once the base score of
all the player dynasties has been calculated, those which are below
the median are given the median base score, and all the scores are
then adjusted by traits and achievements. Note that this means there
is some variation even in the bottom half of the score table; note
further that, if you do very badly on titles but well on achievements
and traits, your final score may be higher than that of someone who
was about average on titles but had no achievements.

\section{Trait Adjustment}

For each dynasty, the
converter counts up the number of members and the numbers of the
traits listed in the \verb|custom_score_traits| object:

\begin{verbatim}
custom_score_traits = {
  inbred = 1
  kinslayer = 1
  ravager = 1
  seaking = 2
  sea_queen = 2
  genius = 1
  berserker = 1
  crusader = 1
  valhalla_bound = 1
  mujahid = 1
  eagle_warrior = 1
  sun_warrior = 1
  ukkos_shield = 1
  nyames_shield = 1
  peruns_chosen = 1
  romuvas_own = 1
  ares_own = 1
  tengri_warrior = 1
  shaddai = 1
  gondi_shahansha = 1
  achievements = 0.01
}
\end{verbatim}

The dynasty's score is then adjusted by the median base score times
the fraction of members with each trait, times the weight of the trait
in the list above. For example, suppose you have 100 dynasts, and two
of them are Sea Kings and one is a Berserker. Then your trait fraction
is 0.05 (two Sea Kings, each worth two points; a Berserker worth one
point; divided by 100 dynasts to get the fraction), and so you will
get a bonus of 5\% of the median base score.

\section{Achievement Adjustment}

Achievement points are listed in the \verb|custom_score| object, which
is created by hand at the end of the CK2 period, when the GM counts up
the achievement score. For example, this listing:
\begin{verbatim}
custom_score = {
  10292922 = 37   # Ivering
  10292920 = 28   # Yngling
}
\end{verbatim}
indicates that there are two player dynasties, with scores of 37 and
28 achievement points respectively. The names on the right are
comments, helpful to the one writing the custom score section but
ignored by the converter code. The achievement points are multiplied
by the \verb|achievements| ``trait''\footnote{Yes, it is a bit of a
kludge to have this all in the same object.} in the custom traits
list; the result is the fraction of the median base score which will
be given as a bonus. For example, the Ynglings, above, with their 28
achievement points, will get 28\% of the median base score as their
achievement adjustment.

\section{Worked Example}

Consider a hypothetical Yngling family with 100 dynasty members
throughout the game. Of these 100, only ten have held landed titles;
to be specific, the ten have held one County, one Duchy, one Kingdom,
and one Empire, and each of them held those titles for ten years. In
addition the first one created the Empire title on day one of the
game, because it is an example. This gives a base score of:
\begin{verbatim}
100 County-years  = 100
100 Duchy-years   = 300
100 Kingdom-years = 500
100 Empire-years  = 700
1 Empire creation =  70
\end{verbatim}
for a total of 1670. Since we are considering only one dynasty, this
is also the median score.

Among the 100 Ynglings, we find a Sea King, a Berserker, a Crusader,
and five Geniuses. The Sea King trait is worth two points, the others
are one point each; the total weighted cool-trait fraction is therefore
0.09 - nine cool-trait points, divided by a hundred dynasts. This is
multiplied by the median base score of 1670 to give a trait adjustment
of 150.3.

Finally, the Ynglings have accumulated 10 achievement points, for a
10\% achievement adjustment, or 167 points. Their total is therefore
1987.3.

\newpage
\appendix{Code}

\lstinputlisting[language=C++,firstline=1337,lastline=1637]{Converter.cc}

\end{document}
